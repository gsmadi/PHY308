\documentclass{article}
\usepackage{geometry,color,graphicx}

\begin{document}

\title{Title template}

\author{Gabriel Smadi\\
  College of Engineering,
  Physics Department, \\
  Syracuse University,\\
  \texttt{gsmadi@syr.edu}}
\maketitle

\begin{abstract}
Abstract goes here.
\end{abstract}

\section{Introduction}
This section's content...

\section{Content}
This section's content...

\begin{equation} \label{eq:1}
x=B\cos(\omega t)
\end{equation}

\begin{equation} \label{eq:2}
v=-\omega B\cos(\omega t)
\end{equation}

\begin{equation} \label{eq:3}
a=-\omega^2 B\cos(\omega t)
\end{equation}

\begin{figure}
  \begin{center}
    \include{position}
  \end{center}
  \caption{Position}
  \label{fig:position}
\end{figure}

Figure \ref{fig:position} shows a graph of the position of a particle. Equation \ref{eq:1} is the position of the particle.

\begin{figure}
  \begin{center}
    \include{velocity}
  \end{center}
  \caption{Velocity}
  \label{fig:velocity}
\end{figure}

Figure \ref{fig:velocity} shows a graph of the velocity of a particle. Equation \ref{eq:2} is the velocity of the particle.

\begin{figure}
  \begin{center}
    \include{acceleration}
  \end{center}
  \caption{Acceleration}
  \label{fig:acceleration}
\end{figure}

Figure \ref{fig:acceleration} shows a graph of the acceleration of a particle. Equation \ref{eq:3} is the acceleration of the particle.

\section{Conclusion}
This section's content...


\begin{thebibliography}{9}

	\bibitem{lamport94}
	  Leslie Lamport,
	  \emph{\LaTeX: A Document Preparation System}.
	  Addison Wesley, Massachusetts,
	  2nd Edition,
	  1994.

\end{thebibliography}
\end{document}
